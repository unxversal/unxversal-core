\documentclass[11pt]{article}
\usepackage[margin=1.5in]{geometry}
\usepackage{amsmath}
\usepackage{amsfonts}
\usepackage{amssymb}
\usepackage{graphicx}
\usepackage{booktabs}
\usepackage{url}

\title{\vspace{2.5in}Systematic Market Making Strategies for Prediction Markets\\[0.5em]\large A Comprehensive Framework for Unxversal Fund}
\author{Unxversal Labs}
\date{\today}

\begin{document}

\maketitle
\thispagestyle{empty}

\newpage
\begin{abstract}
This proposal presents a comprehensive market making framework specifically designed for prediction markets, addressing the unique constraints and opportunities of probability-based trading venues. The strategy adapts traditional Central Limit Order Book (CLOB) market making techniques to handle prices bounded in [0,1], employs log-odds space optimization for superior quote placement near boundaries, and implements robust liability management systems essential for sustainable operations.

The framework encompasses static range making, volatility-adaptive positioning, inventory-aware skewing based on Avellaneda-Stoikov principles, and advanced features including paired mean-reversion strategies and time-to-resolution regime management. Key innovations include systematic cross-venue arbitrage detection, comprehensive risk controls through hard liability caps, and a modular implementation design suitable for institutional deployment.

Expected outcomes include consistent positive carry from bid-ask spread capture, controlled maximum loss scenarios, enhanced returns through adaptive positioning, and reduced correlation with traditional asset classes. The strategy's systematic approach to managing the unique risk profile of prediction markets—where positions have clearly defined maximum loss/payout at resolution—provides Unxversal Fund with a competitive advantage in this rapidly growing asset class.

The technical framework includes real-time liability monitoring, cross-venue connectivity, historical backtesting capabilities, and a comprehensive user interface for strategy configuration and risk management. Implementation follows a phased approach enabling incremental deployment while maintaining strict institutional risk controls.
\end{abstract}
\thispagestyle{empty}

\newpage
\setcounter{page}{1}

\section{Executive Summary}

This proposal outlines a sophisticated market making framework specifically adapted for prediction markets. The strategy ports traditional CLOB (Central Limit Order Book) market making techniques to the unique constraints of prediction markets, where prices represent probabilities bounded in [0,1] and positions have clearly defined maximum loss/payout profiles at resolution.

The framework addresses key challenges including optimal quote spacing in log-odds space, inventory risk management, and liability capping mechanisms essential for sustainable prediction market operations.

\section{Core Framework: CLOB to Prediction Market Mapping}

The fundamental adaptation involves three key modifications from traditional market making:

\begin{itemize}
    \item \textbf{Price Domain:} $p \in [0,1]$ representing probabilities (e.g., YES share at \$0.60 implies 60\% probability)
    \item \textbf{Position Structure:} Clear maximum loss/payout at resolution
    \item \textbf{Quote Optimization:} Superior performance using log-odds space near boundaries
\end{itemize}

\subsection{Risk Profile}
For a position buying YES shares at price $p$:
\begin{align}
\text{Maximum loss per share} &= p \text{ (if resolves NO)} \\
\text{Maximum gain per share} &= 1-p \text{ (if resolves YES)}
\end{align}

\section{Strategy Implementation}

\subsection{Static Range Making in Probability Space}

\textbf{Base Configuration:}
\begin{itemize}
    \item Mid probability: $p_0 = 0.60$
    \item Band: $\pm 0.08 \rightarrow [0.52, 0.68]$
    \item Levels per side: 3 (bids at 0.58, 0.56, 0.54; asks at 0.62, 0.64, 0.66)
    \item Per-level notional: \$2,000
\end{itemize}

\textbf{Liability Caps:}
\begin{itemize}
    \item Maximum net cost (long YES): \$3,000
    \item Maximum net liability from short YES: \$2,000
\end{itemize}

\textbf{Mean-Reversion Cycle Example:}
\begin{enumerate}
    \item Buy 3,448 YES @ 0.58 for \$2,000
    \item Later sell 3,226 YES @ 0.62 for \$2,000
    \item Realized profit: $(0.62-0.58) \times 3,226 \approx \$129$ (minus fees)
    \item Residual inventory: 222 YES shares at 0.58 average cost
\end{enumerate}

\subsection{Log-Odds Space Implementation}

To address asymmetric behavior near probability boundaries, we employ logit transformation:

\begin{align}
\ell &= \text{logit}(p) = \ln\left(\frac{p}{1-p}\right) \\
p &= \text{sigmoid}(\ell) = \frac{1}{1+e^{-\ell}}
\end{align}

\textbf{Example Configuration:}
\begin{itemize}
    \item $p_0 = 0.60 \rightarrow \ell_0 = \ln(0.6/0.4) \approx 0.405$
    \item Step size: $\Delta\ell = 0.1$
    \item Bid levels: $\ell = 0.305, 0.205, 0.105 \rightarrow p \approx 0.576, 0.551, 0.526$
    \item Ask levels: $\ell = 0.505, 0.605, 0.705 \rightarrow p \approx 0.624, 0.647, 0.669$
\end{itemize}

\subsection{Volatility-Adaptive Ladder}

Dynamically adjust quote spacing based on short-term volatility:
\begin{itemize}
    \item Estimate $\sigma(\ell)$ for log-odds volatility
    \item Set band = $\pm 3\sigma$ around $\ell_0$
    \item Step size = $\sigma(\ell)$
    \item Convert back to probability space for order placement
\end{itemize}

\subsection{Inventory-Aware Skewing (Avellaneda-Stoikov Adaptation)}

Implement inventory management through center shifting:
\begin{itemize}
    \item \textbf{Long YES position:} Shift center down (more aggressive selling)
    \item \textbf{Short YES position:} Shift center up (more aggressive buying)
    \item \textbf{Skew rule:} $\delta\ell = k \cdot \text{inventory\_pct}$, capped at $|\delta\ell| \leq 0.15$
\end{itemize}

\section{Advanced Features}

\subsection{Paired Mean-Reversion Strategy}

Upon fill execution, automatically place take-profit orders:
\begin{enumerate}
    \item Buy 3,333 YES @ 0.57 (\$1,900 notional)
    \item Auto-place ask @ 0.585 for 3,333 YES
    \item Expected round-trip profit: $(0.585 - 0.57) \times 3,333 \approx \$50$ per cycle
\end{enumerate}

\subsection{Time-to-Resolution Regime Management}

Implement dynamic risk adjustment based on event timeline:
\begin{itemize}
    \item \textbf{T > 7 days:} Normal band/step parameters
    \item \textbf{T $\in$ [1,7] days:} Widen band by 50\%, reduce size by 30\%
    \item \textbf{T < 24h:} Maker-only far from mid, minimize gap risk
\end{itemize}

\subsection{Cross-Venue Arbitrage}

Exploit structural opportunities:
\begin{itemize}
    \item \textbf{YES + NO Parity:} Ensure $\text{YES\_price} + \text{NO\_price} \leq 1$
    \item \textbf{Multi-venue spread:} Quote around tighter venue, arbitrage via spread trades
    \item \textbf{Categorical markets:} Enforce $\sum \text{outcome prices} \leq 1$
\end{itemize}

\section{Risk Management Framework}

\subsection{Liability Controls}
Define hard limits per strategy:
\begin{itemize}
    \item Maximum net YES cost (long exposure) in \$
    \item Maximum net short YES liability = $\sum(1 - p_{\text{sold}}) \times \text{quantity}$
    \item Maximum per-order size and turnover/hour
    \item Drawdown stop (mark-to-market vs reference mid)
\end{itemize}

\subsection{Bankroll Management Example}
\textbf{Configuration:}
\begin{itemize}
    \item Bankroll: \$10,000
    \item Max net long cost: \$3,000
    \item Max short liability: \$2,000
    \item Per-level notional: \$1,500 (3 levels each side)
\end{itemize}

\textbf{Execution Sequence:}
\begin{enumerate}
    \item Place bids at $p \approx 0.576, 0.551, 0.526$ (\$1,500 each)
    \item Fills at 0.576 and 0.551 (spend \$3,000, hit long-cost cap)
    \item Auto-place TP asks at 0.588 and 0.563
    \item Expected profit: $(\$1,500 \times 0.012) + (\$1,500 \times 0.012) = \$36$ per cycle
    \item Worst-case loss if resolves NO: \$3,000 (known and capped)
\end{enumerate}

\section{Implementation Roadmap}

\subsection{Technical Infrastructure}
\begin{enumerate}
    \item Strategy configuration system with probability/log-odds toggle
    \item Real-time liability monitoring and enforcement
    \item Cross-venue connectivity and arbitrage detection
    \item Historical backtesting framework with resolution-based P\&L simulation
\end{enumerate}

\subsection{User Interface Components}
\begin{itemize}
    \item Price axis editor (Probability vs Log-odds)
    \item Band builder with volatility-adaptive sizing
    \item Inventory management controls with skew parameters
    \item Time regime configuration
    \item Risk cap management interface
    \item Monte Carlo simulation dashboard
\end{itemize}

\section{Expected Performance Metrics}

Based on the framework's systematic approach to capturing bid-ask spreads while managing tail risks through liability caps, we anticipate:

\begin{itemize}
    \item Consistent positive carry from mean-reversion cycles
    \item Controlled maximum loss scenarios through hard liability caps
    \item Enhanced returns from volatility-adaptive positioning
    \item Reduced correlation with traditional asset classes
\end{itemize}

\section{Conclusion}

This prediction market making framework represents a significant advancement in systematic trading strategies for the growing prediction market ecosystem. The combination of probability-aware quote optimization, robust risk management, and adaptive inventory control provides Unxversal Fund with a competitive edge in this emerging asset class.

The strategy's modular design allows for incremental deployment and optimization while maintaining strict risk controls essential for institutional capital preservation.

\end{document}